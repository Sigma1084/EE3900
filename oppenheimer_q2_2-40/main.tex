\documentclass[journal,12pt,twocolumn]{IEEEtran}
%
\usepackage{setspace}
\usepackage{gensymb}
\usepackage{xcolor}
\usepackage{caption}
\singlespacing

\usepackage[cmex10]{amsmath}
\usepackage{mathtools}
\usepackage{hyperref}
\usepackage{amsthm}
\usepackage{mathrsfs}
\usepackage{txfonts}
\usepackage{stfloats}
\usepackage{cite}
\usepackage{cases}
\usepackage{subfig}
\usepackage{longtable}
\usepackage{multirow}
\usepackage{enumitem}
\usepackage{mathtools}
\usepackage{listings}
\let\vec\mathbf

\DeclareMathOperator*{\Res}{Res}
\renewcommand\thesection{\arabic{section}}
\renewcommand\thesubsection{\thesection.\arabic{subsection}}
\renewcommand\thesubsubsection{\thesubsection.\arabic{subsubsection}}

\renewcommand\thesectiondis{\arabic{section}}
\renewcommand\thesubsectiondis{\thesectiondis.\arabic{subsection}}
\renewcommand\thesubsubsectiondis{\thesubsectiondis.\arabic{subsubsection}}

\hyphenation{op-tical net-works semi-conduc-tor}

\lstset{
language=Python,
frame=single, 
breaklines=true,
columns=fullflexible
}


\begin{document}
%

\theoremstyle{definition}
\newtheorem{theorem}{Theorem}[section]
\newtheorem{problem}{Problem}
\newtheorem{proposition}{Proposition}[section]
\newtheorem{lemma}{Lemma}[section]
\newtheorem{corollary}[theorem]{Corollary}
\newtheorem{example}{Example}[section]
\newtheorem{definition}{Definition}[section]
\newcommand{\BEQA}{\begin{eqnarray}}
\newcommand{\EEQA}{\end{eqnarray}}
\newcommand{\define}{\stackrel{\triangle}{=}}
\newcommand{\myvec}[1]{\ensuremath{\begin{pmatrix}#1\end{pmatrix}}}
\newcommand{\mydet}[1]{\ensuremath{\begin{vmatrix}#1\end{vmatrix}}}

\bibliographystyle{IEEEtran}

\providecommand{\nCr}[2]{\,^{#1}C_{#2}} % nCr
\providecommand{\nPr}[2]{\,^{#1}P_{#2}} % nPr
\providecommand{\pr}[1]{\ensuremath{\Pr\left(#1\right)}}
\providecommand{\qfunc}[1]{\ensuremath{Q\left(#1\right)}}
\providecommand{\sbrak}[1]{\ensuremath{{}\left[#1\right]}}
\providecommand{\lsbrak}[1]{\ensuremath{{}\left[#1\right.}}
\providecommand{\rsbrak}[1]{\ensuremath{{}\left.#1\right]}}
\providecommand{\brak}[1]{\ensuremath{\left(#1\right)}}
\providecommand{\lbrak}[1]{\ensuremath{\left(#1\right.}}
\providecommand{\rbrak}[1]{\ensuremath{\left.#1\right)}}
\providecommand{\cbrak}[1]{\ensuremath{\left\{#1\right\}}}
\providecommand{\lcbrak}[1]{\ensuremath{\left\{#1\right.}}
\providecommand{\rcbrak}[1]{\ensuremath{\left.#1\right\}}}

\theoremstyle{remark}
\newtheorem{rem}{Remark}
\newcommand{\sgn}{\mathop{\mathrm{sgn}}}

\providecommand{\abs}[1]{\ensuremath{\left\vert#1\right\vert}}
\providecommand{\res}[1]{\Res\displaylimits_{#1}}
\providecommand{\norm}[1]{\lVert#1\rVert}
\providecommand{\mtx}[1]{\mathbf{#1}}
\providecommand{\mean}[1]{\ensuremath{E\left[ #1 \right]}}
\providecommand{\fourier}{\overset{\mathcal{F}}{ \rightleftharpoons}}
\providecommand{\ztrans}{\overset{\mathcal{Z}}{ \rightleftharpoons}}

\providecommand{\system}{\overset{\mathcal{H}}{ \longleftrightarrow}}
\newcommand{\solution}{\noindent \textbf{Solution: }}
\providecommand{\dec}[2]{\ensuremath{\overset{#1}{\underset{#2}{\gtrless}}}}
\numberwithin{equation}{section}
\makeatletter
\@addtoreset{figure}{problem}
\makeatother

\let\StandardTheFigure\thefigure
\renewcommand{\thefigure}{\theproblem}

\def\putbox#1#2#3{\makebox[0in][l]{\makebox[#1][l]{}\raisebox{\baselineskip}[0in][0in]{\raisebox{#2}[0in][0in]{#3}}}}
\def\rightbox#1{\makebox[0in][r]{#1}}
\def\centbox#1{\makebox[0in]{#1}}
\def\topbox#1{\raisebox{-\baselineskip}[0in][0in]{#1}}
\def\midbox#1{\raisebox{-0.5\baselineskip}[0in][0in]{#1}}

\vspace{3cm}

\title{ Oppenheimer Assignment 2 }
\author{ Sumanth N R }
\maketitle
\tableofcontents

\renewcommand{\thefigure}{\theenumi}
\renewcommand{\thetable}{\theenumi}

\bigskip

\begin{abstract}
	Oppenheim and Schafer Discrete Time Signal 
	Processing Prentise Hall 2nd Edition,
	Solution for Question 2.40
\end{abstract}



\section{Question 2.40}
\noindent Consider a linear time invariant system with impulse response
\begin{align}
	h\sbrak{n} =& \ \brak{\frac{j}{2}}^n u\sbrak{n}
		& \text{where } j = \sqrt{-1}
	\label{eq:q-h-n}
\end{align}
Determine the steady-state response, i.e., the response for large $n$,
to the excitation
\begin{align}
	x\sbrak{n} = \cos\brak{n\pi} u\sbrak{n}
	\label{eq:q-x-n}
\end{align}
\bigskip



\section{Solution}
\noindent We know that,
\begin{align}
	y\sbrak{n} &= h\sbrak{n} * x\sbrak{n} = x\sbrak{n} * h\sbrak{n} \nonumber \\
		&= \sum_{k=-\infty}^{\infty} h\sbrak{k} x\sbrak{n-k} \nonumber \\
		&= \sum_{k=-\infty}^{\infty} \brak{\frac{j}{2}}^{k} u\sbrak{k} \cos\brak{\brak{n-k}\pi} u\sbrak{n-k} \nonumber \\
		&= \sum_{k=0}^{n} \brak{\frac{j}{2}}^k \brak{-1}^{n-k} \nonumber \\
		&= \brak{-1}^n \sum_{k=0}^{n} \brak{\frac{j}{-2}}^k \nonumber \\
	y\sbrak{n} &= \brak{-1}^n \brak{ \frac{1-\brak{-\frac{j}{2}}^{n+1}}{1+\frac{j}{2}} } \label{eq:y-n}
\end{align}
For large $n$, we have,
\begin{align}
	n \to \infty \implies \brak{-\frac{j}{2}}^{n+1} \to 0
	\label{eq:lim-n}
\end{align}
Using \eqref{eq:lim-n} on \eqref{eq:y-n}, we get,
\begin{align}
	y\sbrak{n} &= \brak{-1}^n \brak{ \frac{1}{1+\frac{j}{2}} } \nonumber \\
	\implies y\sbrak{n} &= \frac{\cos\brak{n\pi}}{1 + \frac{j}{2}} \label{eq:y-n-steady}
\end{align}
This, is the steady state response of the system to the excitation given by \eqref{eq:y-n-steady}.



\end{document}
